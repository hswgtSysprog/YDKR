\documentclass[a4paper,10pt]{article}
\usepackage[utf8x]{inputenc}
\usepackage[ngerman]{babel}
\usepackage[T1]{fontenc}      % T1-encoded fonts: auch Wörter mit Umlauten trennen
\usepackage{graphicx} 
\usepackage{fancyvrb}


%opening

\title{Abnahmeprotokoll}
\author{Rainer Hihn 19859\\Kathrin Holzmann 20228\\Florian Rosenkranz 19895}
\date{17.5.2011}
\begin{document}

\maketitle
\newpage
%\begin{abstract}

%\end{abstract}

\section*{Inhalt}

\tableofcontents

\newpage

\section{Abnahme}

Am 17.5.2011 trafen wir uns mit Herrn Elias Drotleff um unseren ersten Zwischenstand abzugeben.\\
Es gab einige Kritikpunkte:\\
\begin{itemize}
       \item Keine Message-Queue, sondern Sockets
       \item Aktives Warten im Client umgehen
       \item Realisierung Phase 1 (Spielvorbereitung):
       \begin{itemize}
	  \item Client-Server Kommunikation (Sockets)
	  \item Login
	  \item Threads im Client und Server
	  \item Loader-Kommunikation (Pipes)
	  \item Mutexe
	\end{itemize}
       \item Realisierung Phase 2(Spiel):
       \begin{itemize}
        \item Shared Mem. erzeugen
	\item Spielablauf (Frage=>Antwort, Timeouts, Punktestand)
	\item ...
       \end{itemize}

     \end{itemize}

\section{Weiteres Vorgehen}

Wir haben auch unser weiteres Vorgehen besprochen. Hierbei werden wir uns insbesondere auf folgende Punkte konzentrieren:\\
\begin{itemize}
 \item Message-Queue entfernen
 \item keygen entfernen
 \item Thread-Start ändern -> nur 3 Threads
 \item Senden und Empfangen nur durch Sockets
 \item Funktionalität entwickeln:
      \begin{itemize}
       \item Login
       \item Playerliste aktualisieren
       \item Fraenkatalog auswählen
       \item Unterschiede in der Behandlung von Spielleiter/Spieler
      \end{itemize}

 \item Playerliste als Array
 \item generell Produktivität/Output steigern
\end{itemize}






\end{document}
