\documentclass[a4paper,10pt]{article}
\usepackage[utf8x]{inputenc}
\usepackage[ngerman]{babel}
\usepackage[T1]{fontenc}      % T1-encoded fonts: auch Wörter mit Umlauten trennen
\usepackage{graphicx} 
\usepackage{fancyvrb}


%opening

\title{Abnahmeprotokoll}
\author{Rainer Hihn 19859\\Kathrin Holzmann 20228\\Florian Rosenkranz 19895}
\date{14.4.2011}
\begin{document}

\maketitle
\newpage
%\begin{abstract}

%\end{abstract}

\section{Inhalt}

\tableofcontents

\newpage

\section{Abnahme}

Am 14.4.2011 trafen wir uns mit Herrn Stefan Gast um unsere ersten Programmentwürfe abzugeben.\\
Es gab ein paar Kritikpunkte:\\
\begin{itemize}
       \item Zu wenig Code
       \item Funktionen in Header-Datei implementiert
     \end{itemize}

\section{Weiteres Vorgehen}

Wir haben auch unser weiteres Vorgehen besprochen. Hierbei werden wir uns insbesondere auf folgende Punkte konzetrieren:\\
\begin{itemize}
 \item Loader im Server implementieren (generell Funktionen aus Header \\entfernen)
 \item Common-Functions einbinden
 \item User-List im Server anlegen
 \item Grafische Oberfläche in Client einbinden
 \item Login-Anfrage stellen
 \item GTK einbinden
 \item Produktivität/Output steigern
\end{itemize}






\end{document}
