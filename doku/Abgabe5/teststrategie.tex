
\documentclass[a4paper,15pt]{article}
\usepackage[utf8x]{inputenc}
\usepackage[ngerman]{babel}
\usepackage[T1]{fontenc}      % T1-encoded fonts: auch Wörter mit Umlauten trennen
\usepackage{graphicx} 
\usepackage{fancyvrb}


%opening

\title{Teststrategie}
\author{Kathrin Holzmann 20228}
\date{22.5.2011}
\begin{document}
\sffamily
\maketitle
\newpage
%\begin{abstract}

%\end{abstract}

\section*{Inhalt}

\tableofcontents

\newpage
\
\section{Teststrategie}

Der Test des Gesamtsystems wird mehrere Phasen durchlaufen.
Zum einen wird ein unbeteiligter Dritter einen Black-Box-Test des Gesamtsystems vornehmen. Durch die Entwickler selbst wird ein White-Box-Test vorgenommen.
\newline
\newline
Dieser gestattet gegenüber dem Black-Box-Test den Blick in den Quellcode und die Entwicklung von
Testwerkzeugen direkt im vorhanden Quellcode.
\newline
\newline
Zum aktuellen Zeitpunkt ist ein Black-Box-Test noch nicht sinnvoll, er wird daher nur vorbereitet und
am Ende der Entwicklungsphase durchgeführt. Der White-Box-Test betrachtet vorerst nur die
vorhandenen Programmmodule und wird dann zur gegebenen Zeit erweitert.
Wird beim White-Box-Test ein Fehler auftreten kann dieser direkt im Quellcode als „To-Do“ markiert
werden. Eine Beschreibung für den Entwickler kann als Kommentar eingefügt werden.
\newpage
\section{Black-Box-Test}
Vorraussetzungen: Zwei Rechner die über ein Netzwerk verbunden sind. Auf einem der PCs sind die Dateien für den Server vorhanden auf dem anderen die für den Client.
\newline

\begin{tabular}{|l|p{5cm}|p{5cm}|l|}
\hline Spielphase & Vorgehen & erwartetes Ergebniss & Testergebniss \\ 
\hline Login 01 & Starten des Clients, ohne den Server vorher zu starten & Fehlermeldung durch den Client ausgegeben. & • \\ 
\hline Login  02 & Server starten, bei den nachfolgenden Testfällen läuft der Server bereits.  Starten des Clients ohne Parameter & Fehlermeldung durch den Client, wegen Fehlender Parameter & • \\ 
\hline Login  03 & Starten des Clients mit angabe des parameters -h & Anzeigen der Hilfe & • \\ 
\hline Login 04 &  Starten des Clients mit angabe des Namens & Spiel vorbereitungsfenster wird angezeigt. Spieler ist Spielleiter.Liste der Kataloge ist angezeigt. Spielername wird lorrekt in der Spielerliste angezeigt & • \\ 
\hline Login 05 & Test Login 04 ist voraussetzung. Weiteren Client anmelden & Spiel vorbereitungsfenster wird angezeigt.Catalogliste wird angezeigt. Spielernamen aller Spieler werden sowohl beim neuen Spieler als auch beim Spielleiter angezeigt. & • \\ 
\hline Login 06 & Test Login 05 ist vorraussetzung. Der Spielleiter wählt einen anderen Katalog aus & Der Spieler bekommt die änderung der Auswahl angezeigt & • \\ 
\hline Login 07 & Der Spielleiter beendet das Spiel & Der Mitspieler bekommt eine Fehlermeldung und das Spiel beendet sich. & • \\ 
\hline Login 08 & Der Spieler beendet das Spiel & Die Benutzerliste des Spielleiters wird aktualisiert. & • \\ 
\hline 
\end{tabular} 
\newpage

Vorraussetzungen: Der Server läuft bereits. Der Spielleiter ist bereits angemeldet.
\newline
\begin{tabular}{|l|p{5cm}|p{5cm}|l|}
\hline Spielphase & Vorgehen & erwartetes Ergebnis & Testergebnis \\ 
\hline Spiel 01 & Der Spielleiter klickt auf starten & Fehlermeldung & • \\ 
\hline Spiel 02 & mindestens ein Weiterer Spieler ist nun angemeldet. Der Spielleiter klickt auf Start & Die erste Frage wird auf allen Clients angezeigt & • \\ 
\hline Spiel 03 & Der Spieler lässt die Zeit ablaufen & Der Spieler bekommt keine Punkte. Der Spieler bekommt eine neue Frage & • \\ 
\hline Spiel 04 & Der Spieler beantwortet die Frage vor ablauf der zeit: richtig & Der Spieler erhält einen Punktestand * verbleibender Zeit und eine neue Frage. & • \\ 
\hline Spiel 05 & Der Spieler beantwortet eine Frage vor ablauf der zeit: Falsch & Der Spiele bekommt keine Punkte und eine neue Frage. & • \\ 
\hline Spiel 06 & Der Spieler verlässt vorzeitig das Spiel & Die Rangliste aller Spieler wird aktualisiert & • \\ 
\hline Spiel 07 & Ein anderer Spieler hat eine Frage richtig beantwortet. & Die Rangliste auf allen Clients wird aktualisiert. & • \\ 
\hline Spiel 08 & Der Spielleiter verlässt das Spiel & Das Spiel wird mit einer Fehlermeldung abgebrochen & • \\ 
\hline Spiel 09 & Ein weiterer Mitspieler versucht sich anzumelden & Der Client erhält eine Fehlermeldung und wird nicht angemeldet & • \\ 
\hline 
\end{tabular} 
\newline
Voraussetzungen: Mehrere Spieler sind angemeldet und unterschiedlich weit im beantworten der Fragen fortgeschritten.
\newline
\begin{tabular}{|l|p{5cm}|p{5cm}|l|}
\hline Spielphase & Vorgehen & erwartetes Ergebnis & Testergebnis \\ 
\hline Ende 01 & Der Spieler hat die letzte Frage beantwortet. & Es erscheint das Ende Fenster. Die Rangliste verändert sich noch & • \\ 
\hline Ende 02 & Der Spieler hat die meisten Punkte & Es erscheint ein Fenster "Sie haben gewonnen" & • \\ 
\hline 
\end{tabular} 

\end{document}
