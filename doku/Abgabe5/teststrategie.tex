
\documentclass[a4paper,15pt]{article}
\usepackage[utf8x]{inputenc}
\usepackage[ngerman]{babel}
\usepackage[T1]{fontenc}      % T1-encoded fonts: auch Wörter mit Umlauten trennen
\usepackage{graphicx} 
\usepackage{fancyvrb}


%opening

\title{Teststrategie}
\author{Rainer Hihn 19859\\Kathrin Holzmann 20228\\Florian Rosenkranz 19895}
\date{22.5.2011}
\begin{document}
\sffamily
\maketitle
\newpage
%\begin{abstract}

%\end{abstract}

\section*{Inhalt}

\tableofcontents

\newpage
\
\section{Teststrategie}

Der Test des Gesamtsystems wird mehrere Phasen durchlaufen.
Zum einen wird ein unbeteiligter Dritter einen Black-Box-Test des Gesamtsystems vornehmen. Durch die Entwickler selbst wird ein White-Box-Test vorgenommen.
\newline
\newline
Dieser gestattet gegenüber dem Black-Box-Test den Blick in den Quellcode und die Entwicklung von
Testwerkzeugen direkt im vorhanden Quellcode.
\newline
\newline
Zum aktuellen Zeitpunkt ist ein Black-Box-Test noch nicht sinnvoll, er wird daher nur vorbereitet und
am Ende der Entwicklungsphase durchgeführt. Der White-Box-Test betrachtet vorerst nur die
vorhandenen Programmmodule und wird dann zur gegebenen Zeit erweitert.
Wird beim White-Box-Test ein Fehler auftreten kann dieser direkt im Quellcode als „To-Do“ markiert
werden. Eine Beschreibung für den Entwickler kann als Kommentar eingefügt werden.
\end{document}
